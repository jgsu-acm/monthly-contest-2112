\documentclass{ctpro}
\usepackage{shortvrb}

\title{ACM算法与微应用实验室2021年12月月赛题目}
\date{2021年12月26日}

\begin{document}
\maketitle
\addproblem{Again! Again!}{1000}{256}{传统}{AgOH}
\addproblem{Base32}{1000}{256}{传统}{Zxilly}

\section*{比赛信息}
\ctinfotab{ACM\ |个人赛|不封榜}{C/C++,Python,Java}{3}

\section*{题目概况}
\problemtab

\section*{编译命令}
参见OJ帮助

\section*{注意事项}
\begin{itemize}
	\item C/C++中函数main()的返回值类型必须是int,程序正常结束时的返回值必须是0。
	\item C/C++代码必须完全符合GNU C/C++ 标准,不能使用诸如绘图、Win32API、中断调用、硬件操作或与操作系统相关的API。
	\item C/C++代码中允许使用STL类库。
\end{itemize}

\paragraph*{} 祝大家取得好成绩!

\MakeShortVerb{\|}
\makeproblem
\section*{题目描述}
分子全为 $1$ 的连分数是一种形如如下形式的数:

$$
	x=a_0
	+\cfrac{1}{a_1
		+\cfrac{1}{a_2
			+\cfrac{1}{a_3
				+\cfrac{1}{\ddots
					+\cfrac{1}{a_n}}}}}
$$

其可以简写为:

$$x=[a_0,a_1,a_2,a_3\dots,a_n]$$

现给定一个连分数 $a_0, a_1, \cdots, a_n$,请你求出它的值对 $998244353$ 取模的结果。

\section*{输入格式}
第一行,一个整数 $n~(1 \leq n \leq {10}^6)$。

第二行,$n$ 个整数 $a_1, a_2, \cdots, a_n~(1 \leq a_i \leq {10}^9)$,代表给定的分子全为 $1$ 的连分数。

\section*{输出格式}
一个整数,代表给定的连分数对 $998244353$ 取模的结果。

数据保证有解。

\section*{输入输出样例}
\testcasetab
{
	3\par
	2 3 3
}
{
	898419920
}

\makeproblem
\section*{题目描述}

Base32 编码是一种使用 32 种字符(字母 A-Z 和数字 2-7)对任意字节数据进行编码的方案,其编码过程如下(以对 |ab| 编码为例):

\begin{enumerate}
	\item 将所给字符串按照字节进行切分(对于 ASCII 字符即转为 ASCII 码):\par
	      |ab| $\rightarrow$ |1100001 1100010|;
	\item 将位数不足 8 位的字节补上前导 0:\par
	      |1100001 1100010| $\rightarrow$ |01100001 01100010|;
	\item 将整个二进制串每 5 位切分成一组:\par
	      |01100001 01100010| $\rightarrow$ |01100 00101 10001 0|;
	\item 最后一组若不足 5 位则在末尾补 0:\par
	      |01100 00101 10001 0| $\rightarrow$ |01100 00101 10001 00000|;
	\item 把得出的若干组二进制数分别转换为十进制数,再通过下表分别转化为对应字符,即为结果:\par
	      |01100 00101 10001 00000| $\rightarrow$ |MFRA|。
\end{enumerate}

\begin{center}
	\begin{tabularx}{\textwidth}{MM|MM|MM|MM}
		\toprule
		\textbf{值} & \textbf{符号} & \textbf{值} & \textbf{符号} & \textbf{值} & \textbf{符号} & \textbf{值} & \textbf{符号} \\
		\midrule
		0           & A             & 8           & I             & 16          & Q             & 24          & Y             \\
		1           & B             & 9           & J             & 17          & R             & 25          & Z             \\
		2           & C             & 10          & K             & 18          & S             & 26          & 2             \\
		3           & D             & 11          & L             & 19          & T             & 27          & 3             \\
		4           & E             & 12          & M             & 20          & U             & 28          & 4             \\
		5           & F             & 13          & N             & 21          & V             & 29          & 5             \\
		6           & G             & 14          & O             & 22          & W             & 30          & 6             \\
		7           & H             & 15          & P             & 23          & X             & 31          & 7             \\
		\bottomrule
	\end{tabularx}
\end{center}

给定一个字符串,请你计算出它的 Base32 编码。

\section*{输入格式}

一行,一个仅包含小写字母的字符串 $S~(1 \leq \vert S \vert \leq 5 \times {10}^6)$。

\section*{输出格式}

一行,输入的字符串的 Base32 编码。

\end{document}
