\documentclass{../cpct/ctpro}
\usepackage{shortvrb}

\title{ACM算法与微应用实验室2021年12月月赛题目}
\date{2021年12月26日}

\begin{document}
\maketitle
\addproblem{Again! Again!}{1000}{256}{传统}{AgOH}
\addproblem{Base! Base!}{1000}{256}{传统}{Zxilly}
\addproblem{Move! Move!}{1000}{256}{传统}{AgOH}
\addproblem{Happy! Happy!}{1000}{128}{传统}{AgOH}
\addproblem{Pair! Pair!}{1000}{256}{传统}{AgOH}
\addproblem{Balloon! Balloon!}{1000}{256}{传统}{AgOH}

\section*{比赛信息}
\ctinfotab{ACM\ |个人赛|不封榜}{C/C++,Python,Java}{3}

\section*{题目概况}
\problemtab

\section*{编译命令}
参见OJ帮助

\section*{注意事项}
\begin{itemize}
    \item C/C++中函数main()的返回值类型必须是int,程序正常结束时的返回值必须是0。
    \item C/C++代码必须完全符合GNU C/C++ 标准,不能使用诸如绘图、Win32API、中断调用、硬件操作或与操作系统相关的API。
    \item C/C++代码中允许使用STL类库。
\end{itemize}

\paragraph*{} 祝大家取得好成绩!

\MakeShortVerb{\|}
\makeproblem
\section*{题目描述}
分子全为 $1$ 的连分数是一种形如如下形式的数:

$$
    x=a_0
    +\cfrac{1}{a_1
        +\cfrac{1}{a_2
            +\cfrac{1}{a_3
                +\cfrac{1}{\ddots
                    +\cfrac{1}{a_n}}}}}
$$

其可以简写为:

$$x=[a_0,a_1,a_2,a_3\dots,a_n]$$

现给定一个连分数 $a_0, a_1, \cdots, a_n$,请你求出它的值对 $998244353$ 取模的结果。

\section*{输入格式}
第一行,一个整数 $n~(1 \leq n \leq {10}^6)$。

第二行,$n$ 个整数 $a_1, a_2, \cdots, a_n~(1 \leq a_i \leq {10}^9)$,代表给定的分子全为 $1$ 的连分数。

\section*{输出格式}
一个整数,代表给定的连分数对 $998244353$ 取模的结果。

数据保证有解。

\section*{输入输出样例}
\testcasetab
{
    3\par
    2 3 3
}
{
    898419920
}

\makeproblem
\section*{题目描述}

Base32 编码是一种使用 32 种字符(字母 A-Z 和数字 2-7)对任意字节数据进行编码的方案,其编码过程如下(以对 |ab| 编码为例):

\begin{enumerate}
    \item 将所给字符串按照字节进行切分(对于 ASCII 字符即转为 ASCII 码):\par
          |ab| $\rightarrow$ |1100001 1100010|;
    \item 将位数不足 8 位的字节补上前导 0:\par
          |1100001 1100010| $\rightarrow$ |01100001 01100010|;
    \item 将整个二进制串每 5 位切分成一组:\par
          |01100001 01100010| $\rightarrow$ |01100 00101 10001 0|;
    \item 最后一组若不足 5 位则在末尾补 0:\par
          |01100 00101 10001 0| $\rightarrow$ |01100 00101 10001 00000|;
    \item 把得出的若干组二进制数分别转换为十进制数,再通过下表分别转化为对应字符,即为结果:\par
          |01100 00101 10001 00000| $\rightarrow$ |MFRA|。
\end{enumerate}

\begin{center}
    \begin{tabularx}{\textwidth}{MM|MM|MM|MM}
        \toprule
        \textbf{值} & \textbf{符号} & \textbf{值} & \textbf{符号} & \textbf{值} & \textbf{符号} & \textbf{值} & \textbf{符号} \\
        \midrule
        0           & A             & 8           & I             & 16          & Q             & 24          & Y             \\
        1           & B             & 9           & J             & 17          & R             & 25          & Z             \\
        2           & C             & 10          & K             & 18          & S             & 26          & 2             \\
        3           & D             & 11          & L             & 19          & T             & 27          & 3             \\
        4           & E             & 12          & M             & 20          & U             & 28          & 4             \\
        5           & F             & 13          & N             & 21          & V             & 29          & 5             \\
        6           & G             & 14          & O             & 22          & W             & 30          & 6             \\
        7           & H             & 15          & P             & 23          & X             & 31          & 7             \\
        \bottomrule
    \end{tabularx}
\end{center}

给定一个字符串,请你计算出它的 Base32 编码。

\section*{输入格式}

一行,一个仅包含小写字母的字符串 $S~(1 \leq \vert S \vert \leq 5 \times {10}^6)$。

\section*{输出格式}

一行,输入的字符串的 Base32 编码。

\section*{输入输出样例}
\testcasetab
{
    qaqqaqqaq
}
{
    OFQXC4LBOFYWC4I
}

\makeproblem
\section*{题目描述}

给定 $1 \sim n$ 的一个排列 $P$ 及一个初始状态下为空的队列 $Q$,每次操作你可以选择 $P$ 中的相邻的两个数字,将其从 $P$ 中移出并按其原来的相对顺序放入 $Q$,你需要一直重复这个操作直到 $P$ 为空。

显然根据决策的不同会导致最终 $Q$ 的结果不同,请求出在所有方案中使得 $Q$ 的字典序最大的方案,并输出此时 $Q$ 的结果。

\section*{输入格式}

第一行,一个整数 $n~(1 \leq n \leq {10}^5)$,代表 $P$ 中数字个数。

第二行,$1 \sim n$ 的一个排列。

\section*{输出格式}

一行,$n$ 个数,代表答案。

\section*{输入输出样例}
\testcasetab
{
    4\par
    1 3 2 4
}
{
    3 2 1 4
}
\testcasetab
{
    4\par
    3 1 4 2
}
{
    4 2 3 1
}

\section*{说明/提示}

【样例解释 \#1】

先从 $P$ 中取出 $(3,2)$ 并放进 $Q$,此时 $Q$ 为 $[3,2]$,再从 $P$ 中取出 $(1,4)$ 放进 $Q$,此时 $Q$ 为 $[3,2,1,4]$。可以证明这是使得 $Q$ 的字典序最大的方案。

【样例解释 \#2】

先从 $P$ 中取出 $(4,2)$ 并放进 $Q$,此时 $Q$ 为 $[4,2]$,再从 $P$ 中取出 $(3,1)$ 放进 $Q$,此时 $Q$ 为 $[4,2,3,1]$。可以证明这是使得 $Q$ 的字典序最大的方案。


\makeproblem
\section*{题目描述}

|Zxilly| 靠着 GSOC 获取了一笔不菲的资金,他准备花掉这笔钱高兴高兴,与众不同的是,|Zxilly| 准备用这笔钱来买一些软件。

|JetBrains| 有 $n$ 个软件,每个软件的售价为 $s_i$。为了高兴,|Zxilly| 决定尽可能多地花掉他的钱;为了高兴,|Zxilly| 可能会重复买同一个软件很多次,然后把多买的授权送给别人。

|AgOH| 偷偷地告诉了你 |Zxilly| 获取的资金金额,你能算出 |Zxilly| “高兴”完后会剩下多少钱吗?

\section*{输入格式}

第一行,一个整数 $M~(1 \leq M \leq {10}^7)$,代表 |Zxilly| 获取的资金金额。

第二行,一个整数 $n~(1 \leq n \leq {10}^4)$,代表 |JetBrains| 有多少种软件。

第三行,$n$ 个整数 $s_i~(1 \leq s_i \leq M)$,代表第 $i$ 种软件的售价为多少。

数据保证 $nM \leq 2 \times {10}^8$。

\section*{输出格式}

一行,一个整数,代表答案。

\section*{输入输出样例}
\testcasetab
{
    17\par
    2\par
    7 9
}
{
    1
}

\makeproblem
\section*{题目描述}

给定一个其中数字各不相同的数列 ${a_n}$,请计算出共有多少个能够使得 $a_i \times a_j = i+j~(i<j)$ 成立的不同的序偶 $(i,j)$。

\section*{输入格式}

第一行,一个整数 $n~(2 \leq n \leq 2 \times {10}^5)$,代表数列 ${a_n}$ 的长度。

第二行,$n$ 个整数 $a_i~(1 \leq a_i \leq 2n)$。

\section*{输出格式}

一行,一个整数,代表答案。

\section*{输入输出样例}
\testcasetab
{
    5\par
    1 3 4 6 5
}
{
    2
}

\makeproblem
\section*{题目描述}

|AgOH| 在参加 ICPC 时获得了很多气球,|AgOH| 想要用这些气球来装点实验室。

这些气球有很多种颜色,看得 |AgOH| 眼花缭乱,然而 |AgOH| 只想要使用一种颜色的气球来装点实验室,而且 |AgOH| 想要在实验室放下尽可能多的气球。请问最终 |AgOH| 会使用哪种颜色的气球来装点实验室?

\section*{输入格式}

第一行,一个整数 $n~(1 \leq n \leq 2 \times {10}^5)$,代表共有多少个气球。

接下来 $n$ 行,每行一个仅由小写字母组成的字符串 $C~(1 \leq \vert C \vert \leq 10)$,代表气球的颜色。

\section*{输出格式}

一个字符串,代表 |AgOH| 会使用哪种颜色的气球来装点实验室。若同时有多种颜色的气球的数量相等,输出颜色的字典序较小的那个。

\section*{输入输出样例}
\testcasetab
{
    5\par
    red\par
    purple\par
    red\par
    blue\par
    purple
}
{
    purple
}

\end{document}
