\documentclass{ctpro}

\title{ACM算法与微应用实验室2021年12月月赛题目}
\date{2021年12月26日}

\begin{document}
\maketitle
\addproblem{Again! Again!}{1000}{256}{传统}{AgOH}

\section*{比赛信息}
\ctinfotab{ACM\ |个人赛|不封榜}{C/C++,Python,Java}{3}

\section*{题目概况}
\problemtab

\section*{编译命令}
参见OJ帮助

\section*{注意事项}
\begin{itemize}
	\item C/C++中函数main()的返回值类型必须是int,程序正常结束时的返回值必须是0。
	\item C/C++代码必须完全符合GNU C/C++ 标准,不能使用诸如绘图、Win32API、中断调用、硬件操作或与操作系统相关的API。
	\item C/C++代码中允许使用STL类库。
\end{itemize}

\paragraph*{} 祝大家取得好成绩!

\makeproblem
\section*{题目描述}
分子全为 $1$ 的连分数是一种形如如下形式的数:

$$
	x=a_0
	+\cfrac{1}{a_1
		+\cfrac{1}{a_2
			+\cfrac{1}{a_3
				+\cfrac{1}{\ddots
					+\cfrac{1}{a_n}}}}}
$$

其可以简写为:

$$x=[a_0,a_1,a_2,a_3\dots,a_n]$$

现给定一个连分数 $a_0, a_1, \cdots, a_n$,请你求出它的值对 $998244353$ 取模的结果。

\section*{输入格式}
第一行,一个整数 $n~(1 \leq n \leq {10}^5)$。

第二行,$n$ 个整数 $a_1, a_2, \cdots, a_n~(1 \leq a_i \leq {10}^5)$,代表给定的分子全为 $1$ 的连分数。

\section*{输出格式}
一个整数,代表给定的连分数对 $998244353$ 取模的结果。

数据保证有解。

\section*{输入输出样例}
\testcasetab
{
	3\par
	2 3 3
}
{
	898419920
}

\end{document}
