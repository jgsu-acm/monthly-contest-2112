\documentclass{ctsol}

\title{ACM算法与微应用实验室2021年12月月赛题解}
\date{2021年12月26日}

\begin{document}
\maketitle
\addsolution{Again! Again!}{AgOH}{数论}
\addsolution{Base! Base!}{Zxilly}{模拟}
\addsolution{Move! Move!}{AgOH}{贪心+链表/线段树}
\addsolution{Happy! Happy!}{AgOH}{完全背包}
\addsolution{Pair! Pair!}{AgOH}{枚举}
\addsolution{Balloon! Balloon!}{AgOH}{模拟}

\section*{题目概览}
\solutiontab

\makesolution
\section*{做法}

按照题意求出连分数的值后进行有理数取余即可得出答案。

因为模数是质数,故可以采取简单的快速幂法求逆元。

\section*{标程}
\std{A}

\makesolution
\section*{做法}

根据题意模拟即可,二进制有关操作可以使用 \verb|bitset| 来轻松搞定。

\section*{标程}
\std{B}

\makesolution
\section*{做法}

\section*{标程}
\std{C}

\makesolution
\section*{做法}

很显然这是一道物品价值与重量相等的完全背包题目,写个板子上去就搞定了。

\section*{标程}
\std{D}

\makesolution
\section*{做法}

枚举 $i+j$ 的值,并判断是否满足条件即可。

\section*{标程}
\std{E}

\makesolution
\section*{做法}

我们需要记录一个字符串出现了多少次,很显然使用一个 \verb|map<string,int>| 就可以把某字符串本身到其出现的次数的映射存下了。

\section*{标程}
\std{F}

\end{document}