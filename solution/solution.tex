\documentclass{ctsol}

\title{ACM算法与微应用实验室2021年12月月赛题解}
\date{2021年12月26日}

\begin{document}
\maketitle
\addsolution{Again! Again!}{AgOH}{数论}
\addsolution{Base! Base!}{Zxilly}{模拟}
\addsolution{Pair! Pair!}{AgOH}{贪心+线段树/链表}
\addsolution{xxx}{AgOH}{模拟}
\addsolution{xxx}{AgOH}{模拟}
\addsolution{xxx}{AgOH}{模拟}

\section*{题目概览}
\solutiontab

\makesolution
\section*{做法}
按照题意求出连分数的值后进行有理数取余即可得出答案。

因为模数是质数,故可以采取简单的快速幂法求逆元。

\section*{标程}
\std{A}

\makesolution
\section*{做法}

根据题意模拟即可,二进制有关操作可以使用 \verb|bitset| 来轻松搞定。

\section*{标程}
\std{B}

\makesolution
\section*{做法}

\section*{标程}
\std{C}

\makesolution
\section*{做法}

很显然这是一道物品价值与重量相等的完全背包题目,写个板子上去就搞定了。

\section*{标程}
\std{D}

\makesolution
\section*{做法}

枚举 $i+j$ 的值,并判断是否满足条件即可。

\section*{标程}

\makesolution
\section*{做法}

\section*{标程}

\end{document}